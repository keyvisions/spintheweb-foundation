% Atto costitutivo della Fondazione Spin the Web
\begin{center}
\fbox{\textbf{Work in Progress – Documento in sviluppo}}
\end{center}

\maketitle

\section*{Premesse}
Le parti di seguito indicate (i \emph{Fondatori}) convengono di costituire una fondazione senza scopo di lucro denominata \definedterm{\FondazioneNome} (di seguito, la \emph{Fondazione}), con sede legale in \FondazioneSedeLegale, avente lo scopo di promuovere e diffondere la cultura dell'\emph{ebranding} e di accompagnare le imprese alla digitalizzazione totale, secondo quanto meglio specificato nello Statuto allegato.

\section*{Costituzione}
\begin{enumerate}[label=\arabic*)]
  \item È costituita una fondazione senza scopo di lucro denominata \definedterm{\FondazioneNome}, di seguito \emph{Fondazione}.
  \item La Fondazione ha sede legale in \definedterm{\FondazioneSedeLegale}. Potranno essere istituite sedi operative o secondarie.
  \item La durata della Fondazione è illimitata.
  \item La Fondazione persegue finalità civiche, solidaristiche e di utilità sociale attraverso la promozione dell'\emph{ebranding} e la trasformazione digitale; le finalità sono dettagliate nello Statuto che, letto, approvato e sottoscritto, costituisce parte integrante del presente atto.
\end{enumerate}

\section*{Dotazione patrimoniale iniziale}
I Fondatori attribuiscono alla Fondazione una dotazione patrimoniale iniziale pari a \definedterm{\PatrimonioIniziale}. La dotazione potrà essere incrementata mediante le fonti di finanziamento indicate nello Statuto: \FontiFinanziamento.

Gli sponsor ricevono consulenza gratuita e altri benefici, modulati in base al livello di sponsorship, come stabilito dal Consiglio di Amministrazione.

I membri del Consiglio di Amministrazione e dell'Organo di Controllo possono contribuire alla sponsorizzazione della Fondazione, nel rispetto delle norme di trasparenza e dei regolamenti interni.

Qualora l'importo devoluto tramite sponsorizzazione superi una soglia definita dal CdA, la somma eccedente verrà investita in prodotti finanziari equilibrati tramite un conto titoli associato al conto bancario della Fondazione, secondo criteri di prudenza e trasparenza.

\section*{Organi e nomine iniziali}
\begin{enumerate}[label=\arabic*)]
  \item Sono sin da ora istituiti gli organi della Fondazione come da Statuto: \OrganiElenco.
  \item I Fondatori procedono alla nomina dei componenti del Consiglio di Amministrazione e del Presidente come da elenco nominativi allegato \textbf{A}.
\end{enumerate}

\section*{Partecipazione e collaborazioni}
La Fondazione accoglie studenti e persone competenti e volenterose nell’argomento, favorendo la partecipazione attiva alle attività e ai progetti. Ha inoltre l’intento di collaborare con università per iniziative di ricerca, formazione e innovazione.

\section*{Efficacia e adempimenti}
\begin{enumerate}[label=\arabic*)]
  \item Il presente atto è soggetto a forma pubblica e verrà ricevuto da Notaio competente; ove richiesto, si procederà alla domanda di riconoscimento della personalità giuridica e, se applicabile, all'iscrizione al RUNTS.
  \item Le spese del presente atto e conseguenti sono a carico della Fondazione.
\end{enumerate}

\vspace{2\baselineskip}
\textbf{Elenco Fondatori}

Inserire elenco dei Fondatori con dati identificativi e firme.

\vspace{2\baselineskip}
\textbf{Allegati}
\begin{itemize}
  \item Allegato A: Elenco componenti organi e accettazioni incarico.
  % Work in Progress – Documento in sviluppo

\section*{Allegato A: Elenco nominativi}

Di seguito l'elenco dei nominativi dei membri del Consiglio di Amministrazione e degli altri organi della Fondazione. I dati dovranno essere completati e verificati prima della presentazione ufficiale.

\begin{itemize}
  \item Presidente: [Nome Cognome]
  \item Segretario: [Nome Cognome]
  \item Consigliere Tecnico: [Nome Cognome]
  \item Organo di Controllo: [Nome Cognome]
  \item (Eventuale) Revisore Legale: [Nome Cognome]
\end{itemize}

% Aggiungere eventuali altri membri o ruoli previsti dallo statuto.

  \item Allegato B: Statuto della Fondazione.
\end{itemize}

  \DataLuogo{\FondazioneCitta}{\today}

\firma{I Fondatori}

% Fine atto costitutivo
