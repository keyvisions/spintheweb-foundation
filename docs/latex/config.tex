% Work in Progress – Documento in sviluppo

% Configurazione comune: denominazione, sede, opzioni

% Opzione ETS (abilitata)
\ETStrue
% \ETSfalse

% Dati base
\newcommand{\FondazioneNome}{Fondazione Spin the Web}
\newcommand{\FondazioneSigla}{FSTW}
\newcommand{\FondazioneSedeLegale}{Via del Carso 2, 36100 Vicenza}
\newcommand{\FondazioneCitta}{Vicenza}
\newcommand{\FondazioneSito}{https://www.spintheweb.org}
\newcommand{\FondazioneEmail}{info@spintheweb.org}

% Missione / scopi sintetici (riassunto)
\newcommand{\FondazioneMissione}{La Fondazione promuove e diffonde la cultura dell'\emph{ebranding} per accompagnare imprese e organizzazioni nella trasformazione digitale integrale, sviluppando standard, formazione, ricerca, progetti di utilità sociale e collaborazioni con università.}

% Patrimonio e risorse
\newcommand{\PatrimonioIniziale}{Euro 50.000,00}
\newcommand{\SogliaCostituzione}{La fondazione viene costituita solo al raggiungimento di una sponsorizzazione minima di Euro 50.000, necessaria a coprire il patrimonio iniziale richiesto e le spese costitutive.}
\newcommand{\FontiFinanziamento}{La fondazione è finanziata esclusivamente tramite sponsorizzazioni. Gli sponsor ricevono, in base al livello di sponsorship, consulenza gratuita e altri benefici definiti dal Consiglio di Amministrazione.}

% Esercizio
\newcommand{\ChiusuraEsercizio}{31 dicembre}

% Organi
\newcommand{\OrganiElenco}{Consiglio di Amministrazione (composto da Presidente, Segretario e Consigliere Tecnico); Organo di Controllo; (eventuale) Revisore Legale}

% Attività e iniziative
\newcommand{\FondazioneAttivita}{%
Seminari:
1. Concetto del eBranding
2. Come strutturarlo
3. Come crearlo

Workshop:
1. Sviluppare portali
2. Integrarsi a portali Spin the Web

Ricerca e Sviluppo:
1. Standard e linee guida
2. Toolkit operativi
3. Progetti pilota con imprese e PA
}
