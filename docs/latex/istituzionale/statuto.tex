% Statuto della Fondazione Spin the Web
\maketitle

\articolo{1 — Denominazione, sede e durata}
\comma{1} È costituita una fondazione senza scopo di lucro denominata \definedterm{\FondazioneNome} (in breve, \FondazioneSigla), con sede legale in \definedterm{\FondazioneSedeLegale}. Il Consiglio di Amministrazione (CdA) può istituire sedi operative o secondarie.
\comma{2} La durata della Fondazione è illimitata.

\articolo{2 — Finalità}
\comma{1} La Fondazione persegue finalità civiche, solidaristiche e di utilità sociale, consistenti nella promozione e diffusione della cultura dell'\emph{ebranding} per accompagnare imprese e organizzazioni nella trasformazione digitale integrale.
\comma{2} In particolare, la Fondazione svolge attività di studio, ricerca, formazione, divulgazione, standardizzazione e supporto a progetti digitali ad impatto sociale, come meglio specificato all'articolo seguente.

\articolo{3 — Attività}
\comma{1} La Fondazione realizza le proprie finalità mediante, a titolo esemplificativo:
\FondazioneAttivita
\begin{enumerate}[label=\alph*)]
  \item ricerca e sviluppo su metodologie e strumenti di ebranding;
  \item programmi di formazione e certificazione delle competenze;
  \item pubblicazioni, linee guida e standard aperti;\footnote{I materiali saranno resi disponibili, ove possibile, in \url{\FondazioneSito}.}
  \item consulenza pro bono o a condizioni agevolate per soggetti meritevoli;
  \item organizzazione di eventi, seminari e comunità professionali;
  \item collaborazione con enti pubblici e privati, università e centri di ricerca;
  \item sviluppo e gestione di piattaforme e progetti digitali coerenti con le finalità istituzionali.
\end{enumerate}
\ifETS
\comma{2} Qualora la Fondazione assuma la qualifica di Ente del Terzo Settore (ETS), le attività di cui al comma 1 sono ricondotte alle tipologie di cui all'art. 5 del D.Lgs. 117/2017 e ad eventuali attività diverse nei limiti dell'art. 6, come deliberate dal CdA e riportate nel bilancio consuntivo e, se dovuto, nel bilancio sociale.
\fi

\articolo{4 — Patrimonio e risorse}
\comma{1} Il patrimonio iniziale è costituito da \definedterm{\PatrimonioIniziale}, oltre ad eventuali ulteriori conferimenti.
\comma{2} Le risorse economiche derivano da: \FontiFinanziamento. 
\comma{3} Gli sponsor ricevono consulenza gratuita e altri benefici, modulati in base al livello di sponsorship, come stabilito dal Consiglio di Amministrazione.
\comma{4} Tutte le transazioni effettuate dalla Fondazione sono tracciate e registrate, nel rispetto della normativa vigente e dei principi di trasparenza.
È vietata la distribuzione, anche indiretta, di utili o avanzi di gestione salvo quanto consentito dalla legge.
\comma{5} I membri del Consiglio di Amministrazione e dell'Organo di Controllo possono contribuire alla sponsorizzazione della Fondazione, nel rispetto delle norme di trasparenza e dei regolamenti interni.

\articolo{5 — Partecipazione e collaborazioni}
\comma{1} La Fondazione accoglie studenti e persone competenti e volenterose nell’argomento, favorendo la partecipazione attiva alle attività e ai progetti. Ha inoltre l’intento di collaborare con università per iniziative di ricerca, formazione e innovazione.
\comma{2} La Fondazione può attivare collaborazioni con altre fondazioni, enti, associazioni e organizzazioni, pubbliche e private, nazionali e internazionali, per il raggiungimento delle proprie finalità.

\articolo{6 — Esercizio finanziario e bilanci}
\comma{1} L'esercizio si chiude al \definedterm{\ChiusuraEsercizio}. Il CdA approva il bilancio consuntivo entro 4 mesi (o 6 in caso di particolari esigenze) dalla chiusura.
\ifETS
\comma{2} In caso di ETS, sono osservati gli obblighi di redazione, deposito e pubblicità dei bilanci previsti dal CTS e dai relativi decreti attuativi; ove ricorrano i presupposti, è predisposto e pubblicato il bilancio sociale.
\fi

\articolo{7 — Organi}
\comma{1} Sono organi della Fondazione: il Consiglio di Amministrazione (CdA), il Presidente, l'eventuale Direttore/Segretario Generale, l'Organo di Controllo e, ove nominato o dovuto, il Revisore Legale.

\articolo{8 — Consiglio di Amministrazione}
\comma{1} Il CdA è composto da tre (3) membri: il Presidente, il Segretario e un Consigliere Tecnico (di seguito, anche ``Tecnico''). La prima nomina è effettuata dai Fondatori; le nomine successive secondo le modalità previste dal presente Statuto e dal Regolamento.
\comma{2} Il CdA dura in carica \textit{(es. 3)} esercizi ed i suoi membri sono rieleggibili. In caso di vacanza di una carica, il CdA provvede tempestivamente alla sostituzione temporanea fino alla successiva nomina secondo Statuto.
\comma{3} Compete al CdA, tra l'altro: definire i piani strategici e annuali, approvare i bilanci, nominare e revocare l'eventuale Direttore/Segretario Generale, deliberare su regolamenti interni, ammettere soggetti sostenitori, decidere su modifiche statutarie e scioglimento nei limiti di legge.

\articolo{9 — Presidente}
\comma{1} Il Presidente rappresenta legalmente la Fondazione, convoca e presiede il CdA, cura l'esecuzione delle deliberazioni e sovrintende alle attività.

\articolo{10 — Direttore/Segretario Generale}
\comma{1} Il CdA può nominare un Direttore/Segretario Generale, al quale possono essere delegati poteri gestionali; partecipa al CdA senza diritto di voto, salvo diversa deliberazione.

\articolo{11 — Organo di Controllo e Revisione}
\comma{1} È istituito un Organo di Controllo monocratico o collegiale nei casi previsti dalla legge o per scelta del CdA; ne sono definiti requisiti, durata e funzioni ai sensi delle norme vigenti.
\ifETS
\comma{2} In caso di ETS e al superamento delle soglie di cui all'art. 31 CTS, è nominato un Revisore Legale o una Società di Revisione per il controllo legale dei conti.
\fi

\articolo{12 — Personale e volontari}
\ifETS
\comma{1} La Fondazione può avvalersi di volontari iscritti in apposito registro, nel rispetto degli artt. 17 e seguenti del CTS; può altresì avvalersi di lavoratori dipendenti o autonomi nei limiti necessari.
\else
\comma{1} La Fondazione può avvalersi di collaboratori, dipendenti e consulenti nel rispetto delle norme vigenti e delle risorse disponibili.
\fi

\articolo{13 — Gratuità delle cariche e rimborsi}
\comma{1} Le cariche degli organi sono, di norma, gratuite; è ammesso il rimborso delle spese documentate e, ove deliberato, un compenso nei limiti di legge.

\articolo{14 — Devoluzione del patrimonio}
\ifETS
\comma{1} In caso di scioglimento, cessazione o estinzione, il patrimonio residuo è devoluto, previo parere positivo dell'Ufficio del RUNTS e salva diversa destinazione imposta dalla legge, ad altri enti del Terzo Settore individuati dal CdA.
\else
\comma{1} In caso di scioglimento, il patrimonio residuo è devoluto secondo legge e secondo le deliberazioni del CdA, sentite le autorità competenti.
\fi

\articolo{15 — Modifiche statutarie}
\comma{1} Le modifiche al presente Statuto sono deliberate dal CdA con le maggioranze qualificate previste dal regolamento e dalla legge.

\articolo{16 — Disposizioni finali}
\comma{1} Per quanto non previsto dal presente Statuto si applicano le norme del codice civile e, ove in vigore e applicabile, del Codice del Terzo Settore.

\vspace{2\baselineskip}
\DataLuogo{\FondazioneCitta}{\today}

% Fine statuto
